\documentclass{article}
\usepackage{graphicx}
\usepackage{amsmath}
%\usepackage[scientific-notation=true]{siunitx}
\usepackage{tabularx}
\newcommand{\D}{\displaystyle}
\begin{document}

\title{Inferring Characteristics of Interaction Matrices in an Ecological Context{}}
\author{Naika Dorilas}

\maketitle

\begin{abstract}
In this paper, 
\end{abstract}

\section{Background}
What is random matrix theory?\hfill\break
Random Matrix Theory is a combination of statistics and traditional matrix theory. In this field, many tools from statitistics are used to describe matrices with random elements instead of relying on those individual elements. It allows us to generalize about certain characteristics of mtrices with similar statistical properties. Many problems in physical systems modeling and biological systems modeling involve matrices with random entries. \hfill\break
Random Matrix Theory in Ecology\hfill\break
-what is it useful for?\hfill\break
Traditionally in one-dimensional models of population growth, specifically lotka volterra there are birth rates $r_b$, death rates $r_d$, and a carrying capacity, K.  \hfill\break
\hfill\break Model \hfill\break \hfill\break
Although, if we want to include the effects of other species on a single species, in ecological models of population growth across species we often include something called an interaction matrix. It describes the way that different species interact with each other and effect eachothers survivial whether it be positively, negatively or with no effect at all. This is an essential aspect of a species environment which plays a large role in its survival. The simplest model of populations size of the ith species in an ecosystem is given by,\hfill\break
\hfill\break 
$y_i(t+1)=y(t) + \Delta t(A_{ij}\times y_i +\xi_i)$
\hfill\break
where $y_i(t)$ is the population of species i at time t, $\xi_i$ is random noise that might affect a species growth, and $A_ij$ is the matrix that describes the interactions between the species.
\hfill\break 
An example interaction matrix is illustrated in Figure (blah)(notes from Andy and page 166 ). 

\hfill\break figure \hfill\break \hfill\break
Here we have a small ecosystem of just 2 species for simplicity. In this case the model for population size of species i can be expanded to, 
\hfill\break
\hfill\break
%\begin{equation}
%\begin{cases}
%\dfrac{dT}{dt}=-\beta V T\\\\
%\dfrac{dI_1}{dt}=\beta V T- kI_1\\\\
%\end{cases}
%\end{equation}
\hfill\break
\hfill\break
(Explain what would happen for a couple different values)\hfill\break
-motivation?\hfill\break
In biological systems many aspects of the environment and growth process are random, large and can vary over space and time. On top of that, the stability of a biological system is very pertinent, in an ecological context it determines the survival of species. Generally we look at the "community matrix", which for a population of S species includes S equations for population growth. The community matrix looks at equilibrium solutions for these equations in small perturbations around the equilibria. We say it is stable if when starting close to an equilibrium we stay close to that equilibria. Local stability can be determined by the eigenvalues of the is matrix M.  For higher dimensions, there is the exact entries of the matrix matter less than certain aspects about the distribution of their entries such as the mean, variance and diagonal elements. Random Matrix Theory allows us to understand the stability of an ecosystem through looking at large scale properties of the matrix A. We can determine when a system will be stable based on these properties\hfill\break
\begin{bmatrix}
   x_{11} & x_{12} & x_{13} & \dots &x_{1, 100}&\dots & x_{1n} \\
    x_{21} & x_{22} & x_{23} & \dots &x_{2,100}&\dots & x_{2n} \\
    \vdots & \vdots & \vdots & \ddots & \vdots &\vdots&\vdots \\
x_{100,1}&x_{100,2}&x_{100,3} &\dots & x_{100,100}&\dots& x_{100,n}\\ 
\vdots & \vdots& \vdots& \vdots& \vdots &\ddots&\vdots \\
    x_{n1} & x_{n2} & x_{n3} & \dots &x_{n,100}&\dots & x_{nn}
\end{bmatrix}
\hfill\break
-relevant resuts from randommatrix theory?\hfill\break
-Maximum likeihood
-graphs that illustrate results\hfill\break
Punchline(what am I trying to prove/show; goal)\hfill\break
In this project we start with a stable symmetrical matrix for which there are many useful properties from Random Matrix Theory. Given a set of data and our population model (1), our goal was to infer paramaters:mean-mu, variance-sigma, diagonal elements-d of our interaction matrix A. What this means is that based on an arbitrary real world data set of populations of different species across time, we hoped to infer how the species in that data set interact with eachother. 



\section{Methods}
Deriving Likelihood\hfill\break
-We want to look at $P(x_i^a|\mu,\sigma, d)$ the probability of observing the data given $\mu,\sigma,d$.
-We do this by ... \hfill\break
Simulating Data\hfill\break
-To ensure that our calculations were correct and that the method for out parameter were valid we generated data from $P(x_i^a| A)=\dfrac{\sqrt{det(A)}^{N/2}}{2\pi^{NS/2}}$ \hfill\break
Testing Likelihood\hfill\break
-To test the likelihood we looked at the 
-graphs\hfill\break
\hfill\break
\subsection{This is a subsection}
Blah
\hfill\break

\hfill\break
%\hfill\break
%\hfill\break
%\begin{figure}
%\caption{Cool graphs}
% \includegraphics[scale=0.25]{}
% \includegraphics[scale=0.25]{}
% \includegraphics[scale=0.25]{}
% \includegraphics[scale=0.25]{}
% \includegraphics[scale=0.25]{}
% \includegraphics[scale=0.25]{}
%\hfill\break
%\end{figure}
%\hfill\break
%\hfill\break

\section{Results}
Predicted output parameters vs actual parameters(include graphs)\hfill\break
Behavior of simulation ouputs\hfill\break
\hfill\break
%\begin{equation}
%\begin{cases}
%\dfrac{dT}{dt}=-\beta V T\\\\
%\dfrac{dI_1}{dt}=\beta V T- kI_1\\\\
%\dfrac{dI_2}{dt}= kI_1- \delta I_2\\\\
%\dfrac{dV}{dt}=\pi I_2- c V\\\\
%\end{cases}
%\end{equation}
%\hfill\break
\hfill\break
In this model,
\hfill\break
\begin{table}
\caption{Cool Table} 
\centering 
\begin{tabular}{lll}
\hline
Variable  & Meaning &Unit\\ [0.5ex]
\hline
\\
&  &  $\D\frac{\mbox{}}{}$ \\ [0.5ex]	
&  &  $\D\frac{\mbox{}}{}$ \\ [0.5ex]
& &$\D\frac{\mbox{}}{}$ \\[0.5ex]
& &  $\D\frac{\mbox{}}{}$  \\ [0.5ex]

\hline
\end{tabular}
\label{table:variables} 
\end{table}
\hfill\break





\section{Discussion}
In this paper,
(basic summary of Methods and Results and contextualize them in terms of the background information)
In further research, 
(future directions)

%%%%%%%%%%%
%CITING
%%%%%%%%%%%

This is a cite \cite{one}

\begin{thebibliography}{999}

\bibitem{one} Gomero, B., \emph{ Latin Hypercube Sampling and Partial Rank Corelation Coefficient Analysis Applied to an Optimal Control Problem,} University of Knoxville Tennessee, .(2012) .


\end{thebibliography}
\end{document}